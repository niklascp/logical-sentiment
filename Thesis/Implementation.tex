%!TEX root = Thesis.tex

\chapter{Implementation}
\label{chap:Implementation}

It was chosen to use the functional programming language \emph{Haskell} for implementing the \emph{proof of concept} program. In the following sections key aspects of the implementation will be presented. For the complete source code for the implementation please see appendix~\ref{code}.

The reason Haskell, specifically the \emph{Glasgow Haskell Compiler}, was chosen as programming language and platform, was the ability to ... 
\clearpage


\section{Tokenizer and tagger}
The tokenizer has a very simple task, namely to convert an input string to a list of tokens (lower case words) that represent the symbols of the language. An example of the transformation is shown in
(\ref{fig:Tokenizer}).
\begin{align}
  &\text{``Put the pyramid onto the table.''} \to 
  \left[ 
  \token{put}, \token{the}, \token{pyramid}, \token{onto}, \token{the}, \token{table} 
  \right] 
  \label{fig:Tokenizer}
\end{align}

