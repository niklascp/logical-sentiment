%!TEX root = Thesis.tex
\chapter{A naive attempt for lexicon acquisition}
\label{chap:brownCorpus}

This appendix describes the efforts that was initially made in order to acquire a CCG lexicon by \emph{transforming} a tagged corpus, namely the \emph{Brown Corpus}. The approach turned out to be very naive, and was dropped in favor for the C\&C models trained on the CCGBank \cite{ccgBank}, in turn The Penn Treebank \cite{pennTreebank}.

\section*{The Brown Corpus}
English is governed by convention rather than formal code, i.e.\ there is no regulating body like the Académie française. Instead, authoritative dictionaries, i.a.\  Oxford English Dictionary, describe usage rather than defining it. Thus in order to acquire a covering lexicon it is necessary to build it from large amount English text.

The Brown Corpus was compiled by \citeauthor{brown} \shortcite{brown} by collecting written works printed in United States during the year 1961. The corpus consists of just over one million words taken from 500 American English sample texts, with the intension of covering a highly representative variety of writing styles and sentence structures.

Notable drawbacks of the Brown Corpus include its age, i.e.\ there are evidently review topics where essential and recurring words used in present day writing was not coined yet or rarely used back 50 years ago. For instance does the Brown Corpus not recognize the words \emph{internet}. However it is one of the only larger \emph{free to use} tagged corpus available, and for this reason is was chosen for the attempt. Even early analysis showed that coverage would be disappointing, since the Brown Copus only contains 80.4\% of the words of the \emph{hotels and restaurants} subset of the \emph{Opinosis data set} \cite{Opinosis} ($\num{3793}$ words). This mean that every 5th word would on average be a guess in the blind. However the approach was continued to see how many sentences would be possible to syntactic analyze with the lexicon.

The corpus is annotated with \emph{part of speech} tags, but does not include the deep structure of a \emph{treebank}. There is a total of 82 different tags, some examples are shown in Table~\ref{tab:brownTags}. As shown from the extract the tags include very limited information, and while some features can be extracted (e.g.\ tense, person) in some cases, the tagging gives no indication of the context.
\begin{table}[ht]
\begin{center}
\begin{tabular}{ll}
  Tag & Description \\ \hline \hline
  VB  & verb, base form \\ \hline
  VBD & verb, past tense \\ \hline
  VBG & verb, present participle/gerund \\ \hline
  VBN & verb, past participle \\ \hline
  VBP & verb, non 3rd person, singular, present \\ \hline
  VBZ & verb, 3rd. singular present 
\end{tabular}
\end{center}
\caption{Extract from the Brown tagging set.}
\label{tab:brownTags}
\end{table}

Without any contextual information the approach for translating this information into lexical categories is becomes very coarse. For instance there are no way of determining whether a verb is \emph{intransitive}, \emph{transitive}, or \emph{di-transitive}. The chosen method was simply to over-generate, i.e. for every verb entries for all three types of verbs was added to the lexicon. In total 62 of such rules were defined, which produced a lexicon containing CCG categories for 84.5\% of $\num{56057}$ unique tokens present in the Brown Corpus.

\section*{Evaluating the lexicon}
To evaluate the coverage of the acquired lexicon an \emph{shift-reduce} parser was implemented in Haskell. It is not the most effective parsing strategy, but was simple to implement and considered efficiently enough to test the lexicon. A representative sample of the \emph{hotels and restaurants} subset of the \emph{Opinosis data set} was selected (156 sentences). The result was that the parser only was able to parse $10.9\%$ of the sentences. The result was very disappointing, and it was not even considered weather these even was correctly parses. Insted it was recognized that building a CCG lexicon from only a tagged corpus is not a feasible approach. Further development of the approach was dropped and instead the component was replaced with the C\&C~tools~\cite{candc}.



